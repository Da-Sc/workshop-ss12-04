\section{Übersetzungs-Phasen}


\subsection{Der Präprozessor}

\begin{frame}{Übersetzungs-Phasen (phases of translation)}
	Es gibt 9 Phasen der Übersetzung von Quellcode zum Programm:
	
	\begin{itemize}
		\item text processing (1-2)
		\item tokenizing (3)
		\item preprocessing (4)
		\item string processing (5-6)
		\item translation (7)
		\item template instantiation (8)
		\item linking (9)
	\end{itemize}
\end{frame}

\begin{frame}[fragile]{Präprozessor}
	Der Präprozessor arbeitet auf der Basis von Tokens, also auf Textausdrücken. Er führt auf dieser Basis folgende Operationen aus:
	
	\begin{itemize}
		\item Einfügen von Dateiinhalten \verb|#include| (diese werden neu übersetzt!)
		\item Suchen \& Ersetzen \verb|#define|
		\item Bedingtes Einfügen \verb|#if|
		\item Abbrechen der Übersetzung \verb|#error|
	\end{itemize}
\end{frame}

\begin{frame}[fragile]{Zusammenhang mit der Übersetzung}
	\begin{itemize}
		\item Der Präprozessor wird durch Direktiven (Anweisungen) innerhalb des Quellcodes programmiert.
		\item Präprozessor-Direktiven stehen in einer eigenen Zeile und beginnen mit einem \verb|#|
		\item Die Quellcode-Dateien dienen zugleich als Eingabe für den Präprozessor. Die Ausgabe geht dann in die nächste Übersetzungsphase.
	\end{itemize}
	
	\pause
	
	Eine Datei ohne Präprozessor-Anweisungen oder auch nur ein Dateiabschnitt, der nicht von Anweisungen betroffen ist, geht unverändert von Eingabe zu Ausgabe.
\end{frame}

\begin{frame}[fragile]{Die include-Direktive}
	\begin{block}{include-Direktive}
		\verb|#include "|\emph{dateiname}\verb|"| \\
		ersetzt diese Zeile durch den Inhalt der Datei \emph{dateiname}. \\
		Der Inhalt der Datei wandert durch die Phasen 1-4 bevor mit dem Preprocessing fortgefahren wird.
	\end{block}
	
	\pause
	\vspace{1em}
	
	Es gibt noch die Variante \verb|#include <|\emph{headername}\verb|>|, diese ist für Header-Dateien aus der Standard-Bibliothek gedacht. Gängige Compiler/Preprocessors behandeln beide Varianten ähnlich, es werden jedoch die Suchpfade für \emph{headername} gegenüber der ersten Variante verändert.
\end{frame}

\begin{frame}{Beispiel: include}
	\footnotesize
	
	\begin{columns}
		\column{0.4\textwidth}
		\emph{my-header.h}
		\lstinputlisting[language=C++, linerange={1-4}]{cpp-code/include-directive.cpp}
		
		\column{0.4\textwidth}
		\lstinputlisting[language=C++, linerange={7-12}]{cpp-code/include-directive.cpp}
	\end{columns}
\end{frame}

\begin{frame}[fragile]{Die define-Direktive}
	\begin{block}{define-Direktive}
		\verb|#define NAME TOKEN0 TOKEN1.....| \\
		Definiert ein (object-like) Macro mit dem Namen \verb|NAME|, die darauf folgenden Token sind optional. Trifft der Präprozessor nach der Definition eines Macros auf das Token \verb|NAME|, so ersetzt er es durch die Token \verb|TOKEN0 TOKEN1....|.
	\end{block}
\end{frame}

\begin{frame}{Beispiel: define-Direktive}
	\footnotesize
	\lstinputlisting[language=C++]{cpp-code/define-directive.cpp}
\end{frame}

\begin{frame}[fragile]{Bedingtes Einfügen}
	\begin{block}{if-, else- und endif-Direktiven}
		\begin{lstlisting}[language=C++]
			#if constant_expression_0
			// (0)
			#else
			// (1)
			#endif
		\end{lstlisting}
		Die Kommentare (0) usw. stehen für jeweils optional mehrere Zeilen Code.
		Dann wird durch den Präprozessor entweder die Zeilen (0) oder (1) in seine Ausgabe (für die weitere Übersetzung) eingefügt. Was eingefügt wird, ist abhängig vom Wert von \verb|constant_expression_0|.
	\end{block}
\end{frame}

\begin{frame}{Beispiel: Bedingtes Einfügen}
	\footnotesize
	\lstinputlisting[language=C++]{cpp-code/conditional-inclusion.cpp}
\end{frame}




\subsection{Der Linker}

\begin{frame}[fragile]{Die translation unit}
	Beim Übersetzen werden bis auf den letzten Schritt die Eingabedateien (.cpp, .cc) unabhängig / getrennt voneinander behandelt.
	
	Hat der Präprozessor alle Einfüge- (\verb|#include|, \verb|#define|) und Auslass-Operationen (\verb|#if|, \verb|#else|) ausgeführt, 
\end{frame}
